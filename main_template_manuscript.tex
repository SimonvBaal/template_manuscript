% This is a template developed by Simon T. van Baal - free to use in accordance with CC-BY license.

\documentclass[12pt, letterpaper]{article}
\usepackage[utf8]{inputenc}
\usepackage{graphicx} % Required for inserting images

\usepackage[english]{babel}

% Makes sure quotation marks go the right way.
\usepackage[autostyle=true]{csquotes}
\MakeOuterQuote{"}

\usepackage{setspace}
\onehalfspacing
\setlength{\parindent}{0pt}
\setlength{\parskip}{2.0ex plus0.5ex minus0.2ex}

\frenchspacing % removes double space after each sentence.

\usepackage{geometry}
\geometry{
  includehead,
  includefoot,
  left=1in,
  right=1in,
  top=0.8in,
  bottom=0.8in,
  headheight=10pt, 
  headsep=0.25in,
  footskip=0.3in}

\usepackage{fancyhdr}

% Tables
\usepackage{longtable}
\usepackage{tabularx}
\usepackage{booktabs}

\usepackage[labelfont=bf, skip = 5pt, font=small]{caption}
\usepackage{subcaption}
\usepackage{graphicx}
\usepackage{authblk}

% Greek symbols in text
\usepackage{textgreek}

\usepackage[backend=biber, style=apa]{biblatex}
\usepackage[hidelinks]{hyperref}

\hypersetup{
    colorlinks=false,
  %  linkcolor=blue,
  %  filecolor=magenta,      
  %  urlcolor=cyan,
  %  pdftitle={Overleaf Example},
  %  pdfpagemode=FullScreen,
    }
\urlstyle{same}

\usepackage[protrusion=true,expansion=true]{microtype} % Improves typography, load after fontpackage is selected

% For generating example text:
\usepackage{lipsum}

% Name the file with references (e.g., from Zotero) "references.bib".
\addbibresource{references.bib}

% Sometimes Latex likes to print notes, this command tells it not to.
\AtEveryBibitem{\clearfield{note}}

% SI command to reorder figure numbers and use "S" as a prefix.
\newcommand{\beginsupplement}{%
        \setcounter{table}{0}
        \renewcommand{\thetable}{S\arabic{table}}%
        \setcounter{figure}{0}
        \renewcommand{\thefigure}{S\arabic{figure}}%
     }


\title{Title}
\author{F. Author\thanks{email address: \texttt{insert@email.com}; Corresponding author}, S. Author, T. Author}
\date{}

\begin{document}

\maketitle

\begin{abstract}
    \noindent \lipsum[1]
\end{abstract}

\section{Introduction}

\lipsum[2]

Using this template, quotation marks work normally, see Table \ref{tab:example}. You can also use Greek letters such as \texttheta{} outside of math mode. 

Normally, the easiest way to add a bibliography file is to add a document in Overleaf and select Zotero. Otherwise, exporting your local machine library will work too.
%Cite as \textcite{} for author (year) citations, but you can also easily add it as a parenthetical citation with \autocite{}. 

% Here is a basic table

\begin{table}[ht]
    \centering
    \caption{Caption}
    \begin{tabular}{ l | c | r | p{3cm}}
        \hline			
        1 & 2 & 3 & "first quote" \\
        4 & 5 & 6 & second line \\
        7 & 8 & 9 & third line \\
        \hline  
    \end{tabular}
    \label{tab:example}
\end{table}

% For more information on how to create tables, visit: https://en.wikibooks.org/wiki/LaTeX/Tables#The_tabularx_package_-_simple_column_stretching.

% Insert figure by deleting \iffalse and \fi. Then refer to a .png or .jpg you upload to the directory.
\iffalse

\begin{figure}
    \centering
    \includegraphics[width=8cm]{figures/}
    \caption{Caption}
    \label{fig:example}
\end{figure}

\fi

\printbibliography
% to print bibliography

\beginsupplement

% If you insert a figure here, it will be Figure S1.

\end{document}
